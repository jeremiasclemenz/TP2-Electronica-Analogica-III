\section{Comparación de resultados}
Se realizó una tabla comparativa de los resultados para luego analizar el origen de las discrepancias entre ellas.

\begin{table}[h]
\centering
\begin{tabular}{|c|c|c|c|c|c|}
\hline
Parámetros & Unidades & Teórico & Simulado & Medido & ERP \\
\hline
\(L\)  & \mu Hy & 2,3 & 2,3 & 2,36 & 10,43\% \\
\hline
\(C_T\)& pF & 56,19 & 53,66 & 53,66 &  \\
\hline
\(f_o\)& MHz & 14 & 14,13 & 14,14 & 1\% \\
\hline
\(BW\)& MHz & 1,4 & 1,1 & 0,44 & 68,57\%\\
\hline
\(Q_c\)&  & 10 & 12,84 & 32,13 & 221,3\%\\
\hline
\(Q_d\)&  & 501,72 & 501,72 & 73,54 & 85,35\% \\
\hline
\(R_p\)& k\Omega  & 101,5 & 101,5 & 14,88 & 85,33\%\\
\hline
\(Z_i_n\)& \Omega  & 50 &  & 33,67 & 32,66\%\\
\hline
\(Z_o_u_t\)& \Omega  & 1000 &  & 191,78 & 80,82\% \\
\hline
\end{tabular}
\caption{Comparación de resultados}
\label{tab:mi_tabla}
\end{table}
Siendo ERP el error relativo porcentual entre el valor teórico y el valor medido, tomando como valor real al teórico.

\newpage
\section{Conclusión}
Se pudo realizar el trabajo practico, en el cual se aplicaron los conceptos vistos en clase como también conceptos vistos en materias anteriores, a su vez, se obtuvieron diferencias entre lo medido y la calculado de manera teórica, a continuación se analizan estas cuestiones:
\begin{itemize}
    \item Se pudo pudo diseñar el inductor de manera teórica y luego implementarlo físicamente, el cual a pesar de presentar discrepancias físicas respecto a su diseño ideal, el valor medido es prácticamente igual al calculado de manera teórica. Sin embargo el inductor no es ideal y sus resistencias de pérdidas \(R_p\) es aproximadamente un 15\% del valor teórico, por lo tanto, la corriente que circulara por esta resistencia será mayor lo cual genera pérdidas y hace menos eficiente el inductor, por lo tanto es de esperar que esto disminuya en igual proporción a \(Q_d\), pero a su vez, si se analiza la ecuación (20) se observa que \(R_L'\) depende de \(R_p\), se reescribe dicha ecuación para poder observar de manera mas intuitiva como afecta \(R_p\):
    \begin{equation}
    R_L' = \frac{2\cdot R_TR_p}{R_p-2R_T} =  \frac{2R_T}{1-2\frac{R_T}{R_p}}
\end{equation}
Es evidente que, lo ideal seria que \(R_p\) tienda a infinito, de esta forma se cumpliría que el paralelo de \(R_a'\) y \(R_L'\) es igual a \(R_T\). Esto afecta directamente en las adaptaciones de salida como en la de entrada, como pudo observarse en las mediciones realizadas. Las capacidades de los instrumentos, conectores, etc. también afectaron directamente en la adaptación de dichas impedancias, como se pudo observar cuando se realizaron dichas mediciones, las capacidades parásitas eran mayores a 100 pF.
    \item La frecuencia de resonancia central \(f_o\) es la esperada, de hecho esta se corresponde con la simulada, la cual tiene en cuenta los capacitores reales utilizados. Sin embargo, el ancho de banda es mucho más restringido de lo que debería, lo que a su vez provoco un \(Q_c\) mucho mayor al calculado. Una manera de forzar a que \(BW\) aumente es conectar una resistencia en paralelo a la bobina, para de esta forma disminuir \(R_T\) y con esto el \(Q_c\)
\end{itemize}

Finalmente, se concluye que teniendo en cuenta lo mencionado anteriormente, se podrían mejorar los resultados del circuito hasta cierto punto, debido a que la bobina se realiza a mano y eso agrega cierto error humano, como también el tiempo que demoraría iterar entre correcciones y como se observó el error que agregan los instrumentos que se disponen, si bien existen formas para mejorar el diseño y el rendimiento del circuito, los resultados obtenidos hasta el momento son satisfactorios dentro del alcance y las limitaciones del proyecto.